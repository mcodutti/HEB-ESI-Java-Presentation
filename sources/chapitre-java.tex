% === Cours de Java
% === Chapitre : Introduction
\section{Et \sigle{Java} dans tout �a ?}
\leconwithtoc

\subsection{Historique}

\begin{frame}[fragile]{Historique de \sigle{Java}}
\begin{itemize}
\item [\emph{92}] \sigle{SUN} cr�e \sigle{oak} (syst�mes embarqu�s).
\\Auteur: James Gosling
\item [\emph{94}] Adapt� � Internet gr�ce aux \emph{applets}. 
\\Devient \sigle{Java}
\item [\emph{96}] Premi�re version stable et gratuite de \sigle{JDK} 
\item [\emph{98}] Sortie de \sigle{Java 2}
\item [\emph{05}] Version \verb|1.5| de \sigle{Java 2}
\item [\emph{09}] \sigle{Oracle} rach�te \sigle{Sun} (et donc \sigle{Java})
\item [\emph{11}] Version \verb|1.7| (Java 7, en GPL) \includegraphics[scale=.5]{../img/java7.jpeg}
\end{itemize}
\end{frame}

\subsection{Les �ditions de \sigle{Java}}

\begin{frame}{Les �ditions de \sigle{Java}}
\emph{Java} est disponible en 3 \emph{�ditions}
  \begin{itemize}
  \item M�me langage
  \item Mais taille de la biblioth�que diff�rente
  \item Adapt� � des situations diff�rentes
  \end{itemize}
\bigskip
\emph{Java SE} (�dition standard)
  \begin{itemize}
  \item Applications monopostes classiques
  \item Les applications Android sont en Java
  \end{itemize}
\end{frame}

\begin{frame}{Les �ditions de \sigle{Java}}
\emph{Java ME} (�dition mobile - plus l�ger) 
  \begin{itemize}
  \item Applications \emph{embarqu�es} : t�l�phones, appareils �lectroniques, \dots
  \item Omnipr�sent
  \end{itemize}
\medskip
\emph{Java EE} (�dition entreprise - plus complet)
  \begin{itemize}
  \item Applications distribu�es : client-serveur, web
  \item Tr�s pr�sent : riche, robuste et portable
  \item Concurrents : \sigle{.NET} (\sigle{Microsoft}), \sigle{PHP}
  \end{itemize}
\end{frame}

\subsection{Pourquoi \sigle{Java} ?}
 
\begin{frame}{Pourquoi \sigle{Java} ?}
R�elles \emph{qualit�s p�dagogiques}  
  \begin{itemize}
  \item Syntaxe claire et pr�cise
  \item Typage fort
  \item D�tection pr�coce des erreurs
  \item Concepts modernes de programmation 
  \item �conomie d'�chelle
  \end{itemize}
\bigskip
A trouv� sa place dans le milieu professionnel
\end{frame}

\begin{frame}{\sigle{Java} et les autres langages}
Vous verrez d'autres langages
  \begin{itemize}
  \item Assembleur en premi�re 
  \item \sigle{C} et \sigle{C++} en deuxi�me 
  \item \sigle{Cobol} en 1�re, 2�me et 3�me (Gestion)
  \end{itemize}
\bigskip
Vous approfondirez Java
  \begin{itemize}
  \item Les ateliers logiciels, applications distribu�es
  \end{itemize}
\end{frame}
