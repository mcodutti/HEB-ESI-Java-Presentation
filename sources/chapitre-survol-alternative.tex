% === Cours de Java
% === Chapitre : Survol 

\section{Alternatives (survol)}

\leconwithabstract{Nous voyons comment traduire les algorithmes contenant des alternatives que vous �crivez en logique}

\begin{frame}[fragile]{Instructions de choix}
Le \emph{Si}
\begin{Java}
  if ( condition ) {
    instructions
  }
\end{Java}
\bigskip
Le \emph{Si-sinon}
\begin{Java}
  if ( condition ) {
    instructions
  } else {
    instructions
  }
\end{Java} 
\end{frame}

\begin{frame}[fragile]{Exemple} 
\begin{Java}
import java.util.Scanner;
public class Test {
  public static void main(String[] args) {
      Scanner clavier = new Scanner(System.in);
      int nombre1;

      nombre1 = clavier.nextInt();
      if (nombre1 < 0) {
         System.out.println(nombre1 + " est n�gatif"); 
      }   
  }
}
\end{Java}
\end{frame}

\begin{frame}[fragile]{Exemple} 
\begin{Java}
import java.util.Scanner;
public class Test {
  public static void main(String[] args) {
      Scanner clavier = new Scanner(System.in);
      int nombre1;

      nombre1 = clavier.nextInt();
      System.out.println(nombre1 + " est un nombre ");
      if (nombre1 < 0) {
         System.out.println("n�gatif"); 
      } else {
         System.out.println("positif");   
      }
  }
}
\end{Java}
\end{frame}

\begin{frame}[fragile]{Exercice} 
Comment traduire cet algorithme ?
\begin{Code}
MODULE Test
    nombre1: Entier
    LIRE nombre1
    SI nombre1 > 0 ALORS
        ECRIRE nombre1, "est positif"
    SINON
        SI nombre1 = 0 ALORS
            ECRIRE nombre1, "est nul"
        SINON
            ECRIRE nombre1, "est n�gatif"
        FIN SI
    FIN SI
FIN MODULE
\end{Code}
\end{frame}

\begin{frame}[fragile]{Expressions bool�ennes} 
Pour les tests, on peut utiliser :
\begin{itemize}
\item Des comparateurs : \java|<|, \java|>|, \java|<=|, \java|>=|, \java|==|, \java|!=|
\item Des op�rateurs bool�ens : \java|&&| (et), \java{||} (ou), \java|!| (non)
\end{itemize} 
\bigskip
\emph{Attention !} Bien distinguer \java|=| et \java|==|
\end{frame}

\begin{frame}[fragile]{Exemple} 
\begin{Java}
import java.util.Scanner;
public class Exemple {
  public static void main(String[] args) {
      Scanner clavier = new Scanner(System.in);
      int nombre1;

      nombre1 = clavier.nextInt();
      if ((nombre1 % 2) == 0) {
         System.out.println("Le nombre est pair");
      } else {
         System.out.println("Le nombre est impair"); 
      }
  }
}
\end{Java}
\end{frame}

\begin{frame}[fragile]{Exemple} 
\begin{Java}
import java.util.Scanner;
public class Exemple {
  public static void main(String[] args) {
      Scanner clavier = new Scanner(System.in);
      int �ge;

      �ge = clavier.nextInt();
      if ( �ge<21 || �ge>=60 ) {
         System.out.println("Tarif r�duit !");
      }
  }
}
\end{Java}
\end{frame}

\begin{frame}[fragile]{Le \og selon-que\fg} 
Premi�re forme
\begin{Java}
  switch(num�roJour) {
    case 1 : intitul�Jour="Lundi"; break;
    case 2 : intitul�Jour="Mardi"; break;
    case 3 : intitul�Jour="Mercredi"; break;
    case 4 : intitul�Jour="Jeudi"; break;
    case 5 : intitul�Jour="Vendredi"; break;
    case 6 : intitul�Jour="Samedi"; break;
    case 7 : intitul�Jour="Dimanche"; break;
    default : intitul�Jour="Inconnu"; break;
  }
\end{Java}
\begin{itemize}
\item Notez le \java{break}
\item Possible avec : entiers, caract�res et chaines
\end{itemize}
\end{frame}

\begin{frame}[fragile]{Le \og selon-que\fg} 
Deuxi�me forme : la logique suivante
\begin{Code}
  Selon que 
    nb > 0 : Ecrire "positif"
    nb = 0 : Ecrire "nul"
    autrement : Ecrire "n�gatif"
  Fin selon que
\end{Code}
s'�crit en Java
\begin{Java}
  if (nb>0) {
      System.out.println("positif");
  } else if (nb==0) {
      System.out.println("nul");
  } else {
      System.out.println("n�gatif");
  }  
\end{Java}
\end{frame}

