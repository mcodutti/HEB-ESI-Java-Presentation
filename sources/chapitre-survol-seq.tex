% === Cours de Java
% === Chapitre : Survol 

\section{Algorithmes s�quentiels (survol)}

\leconwithabstracttoc{Nous voyons comment traduire les algorithmes s�quentiels que vous �crivez au cours de Logique}

\subsection{Structure g�n�rale d'un programme}

\begin{frame}[fragile]{Structure g�n�rale du programme}
\begin{Java}
  public class NomClasse {
    // Mettre ici les modules (on dit m�thode en Java)
  }
\end{Java}
\begin{itemize}
\item Le nom commence par une majuscule
\item Doit se trouver dans le fichier \code|NomClasse.java|
\end{itemize}
\bigskip
\emph{Attention !} En \sigle{Java} : minuscule $\neq$ majuscule
\par\medskip
\emph{Exemple} : on ne peut pas �crire
\begin{Java}
  Public CLASS NomClasse {
    
  }
\end{Java}
\end{frame}

\begin{frame}[fragile]{La m�thode principale}
\begin{Java}
  public class NomClasse {
    public static void main(String[] args) {
      // Code de la m�thode ici     
    }
  }
\end{Java}
\begin{itemize}
\item \java|main| est le nom de la m�thode principale
\item C'est par l� que commence le programme
\item \`A �crire tel quel, on verra pourquoi
\end{itemize}
\end{frame}

\subsection{Variables et assignation}

\begin{frame}[fragile]{Les variables}
Les types disponibles
\begin{center}
\begin{tabular}{r|l}
En Logique & En  Java \\ \hline
Entier & \java|int| \\ 
R�el & \java|double| \\ 
Chaine & \java|String| \\ 
Caract�re & \java|char| \\ 
Bool�en & \java|boolean| \\  
\end{tabular} 
\end{center}
Exemple de d�claration
\begin{Java}
   int nb1;
\end{Java}
\end{frame}

\begin{frame}[fragile]{Les calculs}
L'assignation se fait via le symbole \java|=|
\begin{Java}
   nb1 = 1;
\end{Java}
\bigskip
Les calculs : on dispose de tous les op�rateurs \textit{classiques}
    \\\medskip
    \begin{tabular}{r|l}
    \java|+| & plus \\ 
    \java|-| & moins \\ 
    \java|*| & fois \\ 
    \java|/| & au moins 1 r�el $\Longrightarrow$ division \emph{r�elle} \\
             & 2 entiers $\Longrightarrow$ division \emph{enti�re} (\sigle{DIV} en \sigle{Logique})\\ 
             
    \java|%| & reste (\sigle{MOD} en \sigle{Logique})\\ 
    \end{tabular} 
\end{frame}

\begin{frame}[fragile]{Exemple}
\begin{Java}
  public class Moyenne {
    public static void main(String[] args) {

      int nombre1;
      int nombre2;
      int moyenne;

      nombre1 = 34345;
      nombre2 = -3213213;
      moyenne = (nombre1 + nombre2) / 2;    
      System.out.println(moyenne); 
    }
  }
\end{Java}
\end{frame}

\begin{frame}[fragile]{Exemple}
\begin{Java}
  public class Moyenne {
    public static void main(String[] args) {

      int nombre1 = 34345;
      int nombre2 = -321321;
      double moyenne;

      // division r�elle car un des 2 op�randes est r�el
      moyenne = (nombre1 + nombre2) / 2.0;    
      System.out.println("La moyenne est " + moyenne); 
    }
  }
\end{Java}
\end{frame}

\subsection{Lire au clavier}

\begin{frame}[fragile]{Lire au clavier}
Moins direct que l'affichage � l'�cran
  \begin{itemize}
  \item Applications modernes (graphiques)
  \item Lectures dans des champs de saisie
  \item Parfois utile : test ou apprentissage
  \end{itemize}
\emph{Exemple}
  \begin{Java}
  import java.util.Scanner;
  // ...
  Scanner clavier = new Scanner(System.in);
  // ...
  nombre1 = clavier.nextInt();
  \end{Java}
\end{frame}

\begin{frame}[fragile]{Lire au clavier - Exemple}
\begin{Java}
import java.util.Scanner;

public class Test {
  public static void main(String[] args) {
      Scanner clavier = new Scanner(System.in);
      double nombre1;
      double nombre2;
      double moyenne;

      nombre1 = clavier.nextDouble();
      nombre2 = clavier.nextDouble();
      moyenne = (nombre1 + nombre2) / 2.0;
      System.out.println(moyenne);    
  }
}
\end{Java}
\end{frame}

\begin{frame}{Lire au clavier}
\begin{center}
\begin{tabular}{r|l}
Pour lire\dots & on �crit\dots \\ \hline
un entier & \java|nextInt()| \\ 
un r�el & \java|nextDouble()| \\ 
un bool�en & \java|nextBoolean()| \\ 
un mot & \java|next()|\\  
une ligne & \java|nextLine()|\\  
un caract�re & \java|next().charAt(0)| \\ 
\end{tabular} 
\end{center}
\end{frame}

\subsection{Constantes}

\begin{frame}[fragile]{Constante locale}
Clause \java|final| {} $\Rightarrow$ constante
\par\medskip
Valeur donn�e
  \begin{itemize}
  \item Soit � la d�claration 
  \item Soit par assignation ult�rieure
  \end{itemize} 
\begin{Java}
  final int X = 1;
  final int Y;
  Y = 2*X;
  X = 2; // Erreur : poss�de d�j� une valeur
  Y = 3; // Idem
\end{Java}
Pourquoi une constante au lieu d'un litt�ral ?
\end{frame}

\subsection{Conventions}

\begin{frame}{Conventions de noms}
Pour une variable :
  \begin{itemize}
  \item Tout mettre en \emph{minuscules}
  \item Sauf les d�buts de \textit{noms compos�s} en majuscule
  \end{itemize} 
\medskip
Pour une constante :
  \begin{itemize}
  \item Tout mettre en \emph{majuscules}
  \item Utiliser \java|_|  pour s�parer les mots
  \end{itemize}
\medskip
Dans tous les cas : \emph{�tre explicite}
\\(sauf abr�viations courantes)
\end{frame}

\begin{frame}[fragile]{Conventions de noms}
\emph{Exemples}
\begin{Java}
  String nom;
  int ann�eEtude;
  int nbEtudiants;
  boolean partieFinie;
  final double PI;
  final int TAUX_TVA;   
\end{Java}
\end{frame}

\subsection{Commentaires}

\begin{frame}[fragile]{Le commentaire}
  \begin{Java}
  // Commentaire sur une ligne
  /* Commentaire sur
     plusieurs lignes */
  \end{Java}
\begin{itemize}
\item Destin� aux humains
\item Sans effet sur le programme
\item N�anmoins crucial
\end{itemize}
\end{frame}


